\chapter{Risultati sperimentali}\label{ch:chapter3}

\section{Introduzione}
In questo capitolo verranno illustrati una serie di esperimenti eseguiti per testare le prestazioni del modello creato per gestire la pianificazione dei turni ospedalieri.
In generale, variando il numero di infermieri o il numero dei giorni che compongono il periodo selezionato, si andranno ad analizzare diversi parametri tra cui:
\begin{enumerate}
\item Tempo di calcolo della soluzione ottima;
\item Gap percentuale del problema, ovvero la distanza tra la soluzione ottima corrente trovata e il lower bound;
\item Quali e quanti vincoli vengono violati nei diversi casi.
\end{enumerate}

\subsection{Tempo di esecuzione}

\subsubsection*{Numero di infermieri fissato}

\subsubsection*{Periodo di tempo fissato}

\subsection{Gap percentuale}


\subsection{Vincoli violati}
