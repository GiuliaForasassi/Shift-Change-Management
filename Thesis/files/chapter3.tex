\chapter{Risultati sperimentali}\label{ch:chapter3}

\section{Introduzione}
In questo capitolo verranno illustrati una serie di esperimenti eseguiti per testare le prestazioni del modello creato per gestire la pianificazione dei turni ospedalieri.
In generale, variando il numero di infermieri e il numero dei giorni che compongono il periodo selezionato, si andranno ad analizzare diversi parametri tra cui:
\begin{enumerate}
\item Tempo di calcolo della soluzione ottima;
\item Gap percentuale del problema, ovvero la distanza tra la soluzione ottima corrente trovata e il lower bound;
\item Quali e quanti vincoli vengono violati nei diversi casi.
\end{enumerate}

I dati di input mediante i quali sono stati fatti i seguenti esperimenti sono dati reali, presi dalla documentazione della Competizione degli Infermieri, su cui questa tesi si basa.

\subsection{Tempo di esecuzione}


\subsection{Gap percentuale}


\subsection{Vincoli violati}


\section{Risultati}