\chapter{Conclusioni}\label{ch:conclusioni}
Il problema affrontato è di fondamentale importanza nell'ambito sanitario, e non solo.
Infatti una gestione automatica dei turni lavorativi può aiutare a semplificare la gestione degli orari del personale in qualunque ambito lavorativo, e rimane particolarmente utile soprattutto in ambienti dove è richiesto di gestire un'elevata quantità di personale.

Nonostante l'elevato numero di variabili in gioco e l'elevato numero di vincoli imposti sul problema di ottimizzazione, il modello creato riesce a trovare una buona soluzione ammissibile al problema in tempi molto brevi e accettabili. Si è riusciti, inoltre, a creare un interfaccia funzionante e usabile dal punto di vista dell'utente. 

Per perfezionare il lavoro fatto in questa tesi, si potrebbe ampliare il modello creato con lo scopo di gestire situazioni non trattate, facendo, quindi, in modo che il sistema implementi più funzionalità diventando più generico e completo possibile.

