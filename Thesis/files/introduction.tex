\chapter{Introduzione}\label{ch:introduzione}
Il lavoro di questa tesi è stato costruire un programma per gestire il \textbf{cambio dei turni del personale infermieristico} all'interno di un ospedale.
Il problema preso in esame è da sempre una questione di fondamentale importanza nell'organizzazione del lavoro all'interno della sanità, e anche in altre realtà lavorative. Quindi, lo scopo di questo progetto è stato creare un \textbf{modello di ottimizzazione} per la pianificazione dei turni lavorativi ospedalieri in modo da trovare una soluzione che soddisfi al meglio i vincoli desiderati.

Il lavoro svolto in questa tesi si basa sui requisiti imposti dalla competizione \textit{Second International Nurse Rostering Competition}, avvenuta tra il 2014 e il 2015 con lo scopo di incoraggiare la ricerca sui metodi automatizzati di selezione degli infermieri per risolvere un problema di assegnamento del personale infermieristico.\\

Si è modellato tale problema come un \textbf{problema di programmazione lineare intera}, le cui variabili sono per la maggior parte binarie. 
Tale programma è stato implementato in Python, usando l'ottimizzatore commerciale Gurobi. Inoltre, una volta creato il modello, sono stati fatti degli esperimenti basandosi su dati reali presi direttamente dalle specifiche della competizione.

La tesi è articolata in \textbf{6 capitoli}, questa introduzione è il primo di essi.
Nel secondo capitolo vengono forniti i contenuti teorici che stanno alla base della programmazione lineare intera. Nel terzo capitolo si introduce il problema in questione spiegando i dati che il programma riceve in input e analizzando nel dettaglio i vincoli del modello creato.
Il quarto capitolo mostra i risultati degli esperimenti fatti sui dati reali, con lo scopo di valutare il modello creato.
Successivamente, nel quinto capitolo si mostra l'interfaccia grafica che permette di utilizzare il modello, realizzata mediante il framework Django.
Infine, nell'ultimo capitolo, vengono tratte le conclusioni di questo lavoro di tesi di laurea.



 